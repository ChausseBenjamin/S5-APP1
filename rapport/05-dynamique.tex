\subsection{Moments $\vec{M}_{B,B}$}

Dans cette section-ci, le moment de force $\vec{M}_{B,B}$ est calculé à l'aide des moments trouvés lors de l'analyse statique.
Hors, la somme des moments en dynamique est donnée par l'équation suivante :

\begin{equation*}
  \sum \vec{M}_X = I_X \cdot \alpha
\end{equation*}

En réutilisant les moments de la section de statique, l'équation devient comme suit, où $\vec{M}_{B,B}$ est le couple $C_B$ recherché.
\begin{align}
  I_X \cdot \alpha_{BA} &= \vec{M}_{B,B} + \vec{M}_{A,B} + \vec{M}_{BA,B} \\
  I_X \cdot \alpha_{BA} &= \vec{M}_{B,B} +
    \begin{bmatrix}
      0 \\
      0 \\
      l_2 \cdot m_{A} \cdot g \cdot \cos\varphi
    \end{bmatrix} +
    \begin{bmatrix}
      0 \\
      0 \\
      \frac{l_2}{2} \cdot m_{BA} \cdot g \cdot \cos\varphi
    \end{bmatrix}
\end{align}

L'équation de la somme des moments d'inertie dit qu'une inertie totale doit être obtenue.
Hors, pour $\vec{M}_{BA,B}$ il s'agit d'une masse à une distance $L$ d'un axe de rotation, pour $\vec{M}_{BA,B}$ il s'agit
d'une tige uniforme et pour $M_{B,B}$, aucune inertie car le point est exactement au point choisi pour calculer les moments.

\begin{align}
  I_B &= 0 \\
  I_A &= m \cdot l_2^2 \\
  I_{AB} &= \frac{m \cdot (l_2/2)^2}{3}
\end{align}

Voici la résolution de l'inertie totale :

\begin{align}
  I &= 0 + m_A l_2^2 + \frac{m_{AB} \cdot {l_2}^2}{3} \\
  I &= m_A l_2^2 + \frac{m_{AB} \cdot {l_2}^2}{3} \\
  I &= l_2^2 \cdot \left(m_A^2 + \frac{1}{3}m_{AB}\right)
\end{align}
Une fois remplacé dans l'équation, on peut isoler le moment recherché et calculer tout en fonction des composantes $z$ puisque l'axe est colinéaire.
Voici les démarches :

\begin{align}
  &l_2^2 \cdot \left(m_A^2 + \frac{1}{3}m_{AB}\right) \cdot \alpha_{BA} = \vec{M}_{B,B} +
    \begin{bmatrix}
      0 \\
      0 \\
      l_2 \cdot m_{A} \cdot g \cdot \cos\varphi
    \end{bmatrix} +
    \begin{bmatrix}
      0 \\
      0 \\
      \frac{l_2}{2} \cdot m_{BA} \cdot g \cdot \cos\varphi
    \end{bmatrix} \\
  &-\vec{M}_{{B,B}_z} = l_2 \cdot m_{A} \cdot g \cdot \cos\varphi +
    \frac{l_2}{2} \cdot m_{BA} \cdot g \cdot \cos\varphi -
    l_2^2 \cdot \left(m_A^2 + \frac{1}{3}m_{AB}\right) \cdot \alpha_{BA} \\
  &\vec{M}_{{B,B}_z} = -g \cdot l_2 \cdot \cos\varphi \left(m_{A} +
    \frac{1}{2} \cdot m_{BA}\right) +
    l_2^2 \cdot \left(m_A^2 + \frac{1}{3}m_{AB}\right) \cdot \alpha_{BA}
\end{align}

\subsection{Force $\vec{F}_B$}

En dynamique, la somme des forces externes est :

\begin{equation*}
  \sum\vec{F}_e = m \cdot \vec{\gamma}_G
\end{equation*}

Hors, l'équation de la statique se transforme lorsqu'on veut trouver $\vec{F}_B$ car la somme n'est plus égale à $0$ :

\begin{equation*}
  m \cdot \vec{\gamma}_G = \vec{F}_B +
    \begin{bmatrix}
      0 \\
      g \cdot (m_B + m_{BA} + m_{A}) \\
      0
    \end{bmatrix}
\end{equation*}

L'équation avec $\vec{F}_B$ isolé est :

\begin{equation*}
  \vec{F}_B = m \cdot \vec{\gamma}_G -
    \begin{bmatrix}
      0 \\
      g \cdot (m_B + m_{BA} + m_{A}) \\
      0
    \end{bmatrix}
\end{equation*}

De l'équation plus haut, $\vec{\gamma}_G$ est l'accélération linéaire du centre de masse. $m$ est la masse totale.
$\vec{\gamma}_G$ est exprimé par une accélération tangentielle et une accélération centripète :

\begin{equation*}
  \vec{\gamma}_G = \vec{\gamma}_G^t + \vec{\gamma}_G^n
\end{equation*}

Où :
\begin{itemize}
  \item l'accélération tangentielle est $\alpha \times \vec{r}$
  \item l'accélération centripète est $\vec{\omega} \times (\vec{\omega} \times \vec{r})$
\end{itemize}
$\vec{r}$ est en réalité $\vec{r}_{B/G}$ qui est ceci :

\begin{align}
  \vec{r} &= \frac{l_2}{2}
    \begin{bmatrix}
      \cos\varphi \\
      \sin\varphi \\
      0 \\
    \end{bmatrix} \\
  \vec{r} &=
    \begin{bmatrix}
      \frac{l_2}{2} \cdot \cos\varphi \\
      \frac{l_2}{2} \cdot \sin\varphi \\
      0 \\
    \end{bmatrix}
\end{align}

L'accélération est une constante, posée par la problématique. Et afin de simplifier les équations,
la vitesse initiale est posée comme nulle : $\vec{\omega} = \vec{0}$. Ceci simplifie grandement l'équation.
Une fois remplacé, on obtient :

\begin{align}
  \vec{\gamma}_G &= \vec{\gamma}_G^t + \vec{\gamma}_G^n \\
  \vec{\gamma}_G &= \alpha \times \vec{r} + \vec{\omega} \times (\vec{\omega} \times \vec{r}) \\
  \vec{\gamma}_G &=
    \begin{bmatrix}
      0\\
      0\\
      \alpha_{AB}
    \end{bmatrix}
    \times
    \begin{bmatrix}
      \frac{l_2}{2} \cdot \cos\varphi \\
      \frac{l_2}{2} \cdot \sin\varphi \\
      0 \\
    \end{bmatrix}
    + \vec{0} \times (\vec{0} \times \vec{r}) \\
  \vec{\gamma}_{G_x} &= 0 \cdot 0 + \alpha_{AB} \cdot \frac{l_2}{2} \cdot \sin\varphi \\
  \vec{\gamma}_{G_y} &= \alpha_{AB} \cdot \frac{l_2}{2} \cdot \cos\varphi - 0 \cdot 0 \\
  \vec{\gamma}_{G_z} &= 0 \cdot \frac{l_2}{2} \cdot \sin\varphi + 0 \cdot \frac{l_2}{2} \cdot \cos\varphi \\
  \vec{\gamma}_{G} &=
    \begin{bmatrix}
      \alpha_{AB} \cdot \frac{l_2}{2} \cdot \sin\varphi \\
      \alpha_{AB} \cdot \frac{l_2}{2} \cdot \cos\varphi \\
      0
    \end{bmatrix}
\end{align}
Cela se remplace facilement dans l'équation initiale. Afin de rendre l'équation plus facile à lire, $m_t = m_B + m_{BA} + m_A$.

\begin{align}
  \vec{F}_B &= m_{total} \cdot \vec{\gamma}_G -
    \begin{bmatrix}
      0 \\
      g \cdot (m_B + m_{BA} + m_{A}) \\
      0
    \end{bmatrix} \\
  \vec{F}_B &= m_{t} \cdot
    \begin{bmatrix}
      \alpha_{AB} \cdot \frac{l_2}{2} \cdot \sin\varphi \\
      \alpha_{AB} \cdot \frac{l_2}{2} \cdot \cos\varphi \\
      0
    \end{bmatrix}
    -
    \begin{bmatrix}
      0 \\
      g \cdot m_t \\
      0
    \end{bmatrix} \\
  \vec{F}_B &=
    \begin{bmatrix}
      \alpha_{AB} \cdot m_t \frac{l_2}{2} \cdot \cos\varphi \\
      \alpha_{AB} \cdot m_t \frac{l_2}{2} \cdot \sin\varphi \\
      0 \\
    \end{bmatrix}
    -
    \begin{bmatrix}
      0 \\
      g \cdot m_t \\
      0
    \end{bmatrix} \\
  \vec{F}_B &=
    \begin{bmatrix}
      \alpha_{AB} \cdot m_t \frac{l_2}{2} \cdot \cos\varphi - 0 \\
      \alpha_{AB} \cdot m_t \frac{l_2}{2} \cdot \sin\varphi - g \cdot m_t \\
      0 - 0 \\
    \end{bmatrix} \\
  \vec{F}_B &=
    \begin{bmatrix}
      \alpha_{AB} \cdot m_t \frac{l_2}{2} \cdot \cos\varphi \\
      (\alpha_{AB} \cdot m_t \frac{l_2}{2} \cdot \sin\varphi) - g \cdot m_t \\
      0 \\
    \end{bmatrix}
\end{align}
