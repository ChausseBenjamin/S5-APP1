\subsection{Moments \vec M_{B,B}}
Dans cette section-ci, le moment de force \vec M_{B,B} est calculé à l'aide des moments trouvés lors de l'analyse statique.
Hors, la somme des moments en dynamique est données par l'équation suivante:

\begin{equation*}
\sum \vec M=I_X\cdot\alpha
\end{equation*}
En réutilisant les moments de la section de statique, l'équation devient comme suit, ou \vec M_{B,B} est le couple C_B recherché.
\begin{align}
I_X\cdot\alpha_{BA}=\vec M_{B,B}+\vec M_{A,B}+\vec M_{BA,B}

I_X\cdot\alpha_{BA}=
\vec M_{B,B}+
\begin{bmatrix}
0 \\
0 \\
l_2\cdot m_{A}\cdot g\cdot\cos\varphi
\end{bmatrix}
+
\begin{bmatrix}
0 \\
0 \\
\frac{l_2}{2}\cdot m_{BA}\cdot g\cdot\cos\varphi
\end{bmatrix}
\end{align}

L'équation de la somme des moments d'inertie dit que qu'une inertie total doit être obtenue.
Hors, pour \vec M_{BA,B} il s'agit d'une masse à une distance L d'un axe de rotation, pour \vec M_{BA,B} il s'agit
d'une tige uniforme et pour M_{B,B}, aucune inertie car le point est exactement au point choisi pour calculé les moments.
\begin{align}
I_B=0
I_A=m\cdot l_2^2
I_{AB}=\frac{m\cdot (l_2/2)^2}{3}
\end{align}
Voici la résolution de l'inertie total:
\begin{align}
I=0+m_Al_2^2+\frac{m_{AB}\cdot{l_2}^2}{3}

I=m_Al_2^2+\frac{m_{AB}\cdot{l_2}^2}{3}

I=l_2^2\cdot (m_A^2+\frac{1}{3}m_{AB})
\end{align}
Une fois remplacé dans l'équation, on peut isolé le moment rechercher et calculé tout en fonction des composantes $z$ puisque l'axe est colinéaire.
Voici les démarches:
\begin{align}
l_2^2\cdot (m_A^2+\frac{1}{3}m_{AB})
\cdot\alpha_{BA}
=
\vec M_{B,B}+
\begin{bmatrix}
0 \\
0 \\
l_2\cdot m_{A}\cdot g\cdot\cos\varphi
\end{bmatrix}
+
\begin{bmatrix}
0 \\
0 \\
\frac{l_2}{2}\cdot m_{BA}\cdot g\cdot\cos\varphi
\end{bmatrix}

-\vec M_{{B,B}_z}
=
l_2\cdot m_{A}\cdot g\cdot\cos\varphi
+
\frac{l_2}{2}\cdot m_{BA}\cdot g\cdot\cos\varphi
-l_2^2\cdot (m_A^2+\frac{1}{3}m_{AB})
\cdot\alpha_{BA}

\vec M_{{B,B}_z}
=
-g\cdot l_2\cdot\cos\varphi(m_{A}
+
\frac{1}{2}\cdot m_{BA})
+l_2^2\cdot (m_A^2+\frac{1}{3}m_{AB})
\cdot\alpha_{BA}
\end{align}

\subsection{Force \vec F_B}

En dynamique, la somme des force externes c'est

\begin{equation*}
\sum\vec F_e=m\cdot\vec\gamma_G
\end{equation*}

Hors, l'équation de la statique se transforme lorsqu'on veut trouvé $\vec F_B$ car la somme n'est plus égal à $0$:

\begin{equation*}
m\cdot\vec\gamma_G=
\vec F_B +
\begin{bmatrix}
0 \\
g\cdot(m_B+m_{BA}+m_{A}) \\
0
\end{bmatrix}
\end{equation*}

L'équation avec $\vec F_B$ isolé est:

\begin{equation*}
\vec F_B =
m\cdot\vec\gamma_G-
\begin{bmatrix}
0 \\
g\cdot(m_B+m_{BA}+m_{A}) \\
0
\end{bmatrix}
\end{equation*}

De l'équation plus haut, $\vec\gamma_G$ c'est l'accélération linéaire du centre de masse. $m$ c'est la masse total.
$\vec\gamma_G$ est exprimé par une accélération tangentielle et une accélération centripète:


\begin{equation*}
\vec\gamma_G=\vec\gamma_G^t+\vec\gamma_G^n
\end{equation*}

Ou:
	- l'accélération tangentielle c'est $\alpha\times\vec r$
	- l'accélération centripète c'est $\vec\omega\times(\vec\omega\times\vec r)$
$\vec r$ est en réalité $\vec r_{B/G}$ qui est ceci:

\begin{align}
\vec r=\frac{l_2}{2}
\begin{bmatrix}
\cos\varphi \\
\sin\varphi \\
0 \\
\end{bmatrix}

\vec r=
\begin{bmatrix}
\frac{l_2}{2}\cdot\cos\varphi \\
\frac{l_2}{2}\cdot\sin\varphi \\
0 \\
\end{bmatrix}
\end{align}

L'accélération est une constante, posé par la problématique. Et afin de simplifé les équations,
La vitesse initial est posé comme nulle: $\vec\omega=\vec0$. Ceci simplifie grandement l'équation.
Une fois remplacé, on obtient:

\begin{align}
\vec\gamma_G=\vec\gamma_G^t+\vec\gamma_G^n


\vec\gamma_G=\alpha\times\vec r+\vec\omega\times(\vec\omega\times\vec r)

\vec\gamma_G=
\begin{bmatrix}
0\\
0\\
\alpha_{AB}
\end{bmatrix}
\times
\begin{bmatrix}
\frac{l_2}{2}\cdot\cos\varphi \\
\frac{l_2}{2}\cdot\sin\varphi \\
0 \\
\end{bmatrix}
+
\vec0\times(\vec0\times\vec r)


\vec\gamma_{G_x}=0\cdot0+\alpha_{AB}\cdot\frac{l_2}{2}\cdot\sin\varphi


\vec\gamma_{G_y}=\alpha_{AB}\cdot\frac{l_2}{2}\cdot\cos\varphi - 0\cdot0


\vec\gamma_{G_z}=0\cdot\frac{l_2}{2}\cdot\sin\varphi +0\cdot\frac{l_2}{2}\cdot\cos\varphi


\vec\gamma_{G}=
\begin{bmatrix}
\alpha_{AB}\cdot\frac{l_2}{2}\cdot\sin\varphi \\
\alpha_{AB}\cdot\frac{l_2}{2}\cdot\cos\varphi \\
0
\end{bmatrix}
\end{align}
Ça se remplace facilement dans l'équation initial. Afin de rendre l'équation plus facile à lire; $m_t=m_B+m_{BA}+m_A$.
\begin{align}
\vec F_B =
m_{total}\cdot\vec\gamma_G-
\begin{bmatrix}
0 \\
g\cdot(m_B+m_{BA}+m_{A}) \\
0
\end{bmatrix}

\vec F_B =
m_{t}\cdot
\begin{bmatrix}
\alpha_{AB}\cdot\frac{l_2}{2}\cdot\sin\varphi \\
\alpha_{AB}\cdot\frac{l_2}{2}\cdot\cos\varphi \\
0
\end{bmatrix}
-
\begin{bmatrix}
0 \\
g\cdot m_t \\
0
\end{bmatrix}

\vec F_B =
\begin{bmatrix}
\alpha_{AB}\cdot m_t\frac{l_2}{2}\cdot\cos\varphi \\
\alpha_{AB}\cdot m_t\frac{l_2}{2}\cdot\sin\varphi \\
0 \\
\end{bmatrix}
-
\begin{bmatrix}
0 \\
g\cdot m_t \\
0
\end{bmatrix}

\vec F_B =
\begin{bmatrix}
\alpha_{AB}\cdot m_t\frac{l_2}{2}\cdot\cos\varphi-0 \\
\alpha_{AB}\cdot m_t\frac{l_2}{2}\cdot\sin\varphi-g\cdot m_t \\
0-0 \\
\end{bmatrix}

\vec F_B =
\begin{bmatrix}
\alpha_{AB}\cdot m_t\frac{l_2}{2}\cdot\cos\varphi \\
(\alpha_{AB}\cdot m_t\frac{l_2}{2}\cdot\sin\varphi)-g\cdot m_t \\
0 \\
\end{bmatrix}
\end{align}