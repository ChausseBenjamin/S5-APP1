
\subsection{Force \vec F_B}

En dynamique, la somme des force externes c'est

\begin{equation*}
\sum\vec F_e=m\cdot\vec\gamma_G
\end{equation*}

Hors, l'équation de la statique se transforme lorsqu'on veut trouvé $\vec F_B$ car la somme n'est plus égal à $0$:

\begin{equation*}
m\cdot\vec\gamma_G=
\vec F_B +
\begin{bmatrix}
0 \\
g\cdot(m_B+m_{BA}+m_{A}) \\
0
\end{bmatrix}
\end{equation*}

L'équation avec $\vec F_B$ isolé est:

\begin{equation*}
\vec F_B =
m\cdot\vec\gamma_G-
\begin{bmatrix}
0 \\
g\cdot(m_B+m_{BA}+m_{A}) \\
0
\end{bmatrix}
\end{equation*}

De l'équation plus haut, $\vec\gamma_G$ c'est l'accélération linéaire du centre de masse. $m$ c'est la masse total.
$\vec\gamma_G$ est exprimé par une accélération tangentielle et une accélération centripète:


\begin{equation*}
\vec\gamma_G=\vec\gamma_G^t+\vec\gamma_G^n
\end{equation*}

Ou:
	- l'accélération tangentielle c'est $\alpha\times\vec r$
	- l'accélération centripète c'est $\vec\omega\times(\vec\omega\times\vec r)$
$\vec r$ est en réalité $\vec r_{B/G}$ qui est ceci:

\begin{align}
\vec r=\frac{l_2}{2}
\begin{bmatrix}
\cos\varphi \\
\sin\varphi \\
0 \\
\end{bmatrix}

\vec r=
\begin{bmatrix}
\frac{l_2}{2}\cdot\cos\varphi \\
\frac{l_2}{2}\cdot\sin\varphi \\
0 \\
\end{bmatrix}
\end{align}

L'accélération est une constante, posé par la problématique. Et afin de simplifé les équations,
La vitesse initial est posé comme nulle: $\vec\omega=\vec0$. Ceci simplifie grandement l'équation.
Une fois remplacé, on obtient:

\begin{align}
\vec\gamma_G=\vec\gamma_G^t+\vec\gamma_G^n


\vec\gamma_G=\alpha\times\vec r+\vec\omega\times(\vec\omega\times\vec r)

\vec\gamma_G=
\begin{bmatrix}
0\\
0\\
\alpha_{AB}
\end{bmatrix}
\times
\begin{bmatrix}
\frac{l_2}{2}\cdot\cos\varphi \\
\frac{l_2}{2}\cdot\sin\varphi \\
0 \\
\end{bmatrix}
+
\vec0\times(\vec0\times\vec r)


\vec\gamma_{G_x}=0\cdot0+\alpha_{AB}\cdot\frac{l_2}{2}\cdot\sin\varphi


\vec\gamma_{G_y}=\alpha_{AB}\cdot\frac{l_2}{2}\cdot\cos\varphi - 0\cdot0


\vec\gamma_{G_z}=0\cdot\frac{l_2}{2}\cdot\sin\varphi +0\cdot\frac{l_2}{2}\cdot\cos\varphi


\vec\gamma_{G}=
\begin{bmatrix}
\alpha_{AB}\cdot\frac{l_2}{2}\cdot\sin\varphi \\
\alpha_{AB}\cdot\frac{l_2}{2}\cdot\cos\varphi \\
0
\end{bmatrix}
\end{align}
Ça se remplace facilement dans l'équation initial. Afin de rendre l'équation plus facile à lire; $m_t=m_B+m_{BA}+m_A$.
\begin{align}
\vec F_B =
m_{total}\cdot\vec\gamma_G-
\begin{bmatrix}
0 \\
g\cdot(m_B+m_{BA}+m_{A}) \\
0
\end{bmatrix}

\vec F_B =
m_{t}\cdot
\begin{bmatrix}
\alpha_{AB}\cdot\frac{l_2}{2}\cdot\sin\varphi \\
\alpha_{AB}\cdot\frac{l_2}{2}\cdot\cos\varphi \\
0
\end{bmatrix}
-
\begin{bmatrix}
0 \\
g\cdot m_t \\
0
\end{bmatrix}

\vec F_B =
\begin{bmatrix}
\alpha_{AB}\cdot m_t\frac{l_2}{2}\cdot\cos\varphi \\
\alpha_{AB}\cdot m_t\frac{l_2}{2}\cdot\sin\varphi \\
0 \\
\end{bmatrix}
-
\begin{bmatrix}
0 \\
g\cdot m_t \\
0
\end{bmatrix}

\vec F_B =
\begin{bmatrix}
\alpha_{AB}\cdot m_t\frac{l_2}{2}\cdot\cos\varphi-0 \\
\alpha_{AB}\cdot m_t\frac{l_2}{2}\cdot\sin\varphi-g\cdot m_t \\
0-0 \\
\end{bmatrix}

\vec F_B =
\begin{bmatrix}
\alpha_{AB}\cdot m_t\frac{l_2}{2}\cdot\cos\varphi \\
(\alpha_{AB}\cdot m_t\frac{l_2}{2}\cdot\sin\varphi)-g\cdot m_t \\
0 \\
\end{bmatrix}
\end{align}
