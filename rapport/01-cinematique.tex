\subsection{Analyse géométrique}
\label{sec:geometry}

\noindent
\begin{minipage}[t]{0.4\textwidth}
\begin{figure}[H]
  \centering
  \begin{center}
    \includegraphics[width=\textwidth]{figures/geometry.pdf}
  \end{center}
  \caption{Tiges du bras mécanique}
  \label{fig:geometry}
\end{figure}
\end{minipage}
\hfill
\begin{minipage}[t]{0.6\textwidth}

\begin{align}
  \vec{XO} = \begin{bmatrix} 0\\l_0\\0 \end{bmatrix} \hspace{.5cm}
  \vec{OB} &= \begin{bmatrix} l_1\cos{\theta}\\ l_1\sin{\theta} \\0 \end{bmatrix} \hspace{.5cm}
  \vec{BA} = \begin{bmatrix} l_2\cos{\varphi}\\l_2\sin{\varphi} \\0 \end{bmatrix} \\
  \vec{B} = \vec{XO}+\vec{OB} &= \begin{bmatrix} l_1\cos{\theta}\\l_0+l_1\sin{\theta} \\0 \end{bmatrix} \\
  \vec{A} = \vec{B}+\vec{BA} &= \begin{bmatrix}
  l_1\cos{\theta}+l_2\cos{\varphi}\\l_0+l_1\sin{\theta}+l_2\sin{\varphi} \\0 \end{bmatrix} \\
  \intertext{Autrement dit:}
  B_x(\theta) &= l_1\cos{\theta} \\
  B_y(\theta) &= l_0+l_1\sin{\theta} \\
  \label{eq:gen-a-x}
  A_x(\theta,\varphi) &= l_1\cos{\theta}+l_2\cos{\varphi} \\
  A_y(\theta,\varphi) &= l_0+l_1\sin{\theta}+l_2\sin{\varphi} \\
\end{align}
\end{minipage}

\subsection{Vitesses}
\label{sec:speed}

Lorsque des angles fixes sont considérés, il est possible de trouver les coordonnées
$x$, $y$ en fonction de ceux-ci. Toutefois, il se peut que ces angles varient
dans le temps (essentiel à considérer pour déterminer la vitesse du
point $A$). $\theta$ et $\varphi$ sont donc dorénavant présentés comme des fonctions
pour le reste de l'analyse cinématique générale du bras mécanique :

\begin{equation*}
  \theta \rightarrow \theta(t) &\hspace{1cm} \varphi \rightarrow \varphi(t)
\end{equation*}

Déterminer la vitesse du point $A$ ne devient par la suite qu'un exercice de
dérivation dans le temps de la position, en utilisant les équations développées
à la section \ref{sec:geometry} comme fondement. Afin d'en faciliter la résolution (et
la lecture), cette opération est segmentée par composantes $x$ et $y$ dans les
sections suivantes.

\subsubsection{Composante $x$}

\begin{align}
  A_x(t) &= l_1\cos{\theta(t)}+l_2\cos{\varphi(t)} \hspace{1cm} \\
  \dd{t} A_x(t) &= \dd{t} l_1\cos(\theta(t)) + \dd{t} l_2\cos(\varphi(t)) \\
  V_A_x(t) &= \dd{t} l_1\cos(\theta(t)) + \dd{t} l_2\cos(\varphi(t)) \\
  V_A_x(t) &= l_1\dd{t} \cos(\theta(t)) + l_2\dd{t} \cos(\varphi(t)) \\
  V_A_x(t) &= l_1\left( -\sin(\theta(t))\dot\theta(t)  \right) +
    l_2\left( - \sin(\varphi(t))\dot\varphi(t) \right) \\
  V_A_x(t) &= -l_1\sin(\theta(t))\dot\theta(t)
    - l_2\sin(\varphi(t))\dot\varphi(t)
\end{align}

\subsubsection{Composante $y$}
\begin{align}
  A_y(t)&=l_0+l_1\sin(\theta(t))+l_2\sin(\varphi(t)) \\
  \dd{t}A_y(t) &=
      \dd{t}l_0
    + \dd{t}l_1\sin(\theta(t))
    + \dd{t}l_2\sin(\varphi(t)) \\
  V_{A_y}(t) &=
      \dd{t}l_1\sin(\theta(t))
    + \dd{t}l_2\sin(\varphi(t)) \\
  V_{A_y}(t) &=
      l_1\dd{t}\sin(\theta(t))
    + l_2\dd{t}\sin(\varphi(t)) \\
  V_{A_y}(t) &=
      l_1\cos(\theta(t))\cdot\dot{\theta}(t)
    + l_2\cos(\varphi(t))\cdot\dot{\varphi}(t)
\end{align}

\subsection{Accélérations}

L'accélération se définit comme étant la dérivée de la vitesse. Comme la
vitesse a déjà été définie de façon générale à la section \ref{sec:speed}, il ne suffit
qu'à en effectuer la dérivée (encore une fois par composante pour simplifier
la lecture) afin d'obtenir une solution générale de l'accélération selon
n'importe quelles fonctions décrivant le mouvement angulaire des moteurs
dans le temps.

\subsubsection{Composante $x$}

\begin{align}
  V_A_x(t) &= -l_1\sin(\theta(t))\dot\theta(t)
    - l_2\sin(\varphi(t))\dot\varphi(t) \\
  \dd{t} V_A_x(t) &= \dd{t}\left(-l_1\sin(\theta(t))\dot\theta(t)
    - l_2\sin(\varphi(t))\dot\varphi(t)\right) \\
  \alpha_A_x(t) &= -l_1 \dd{t}\sin(\theta(t))\dot\theta(t)
    - l_2\dd{t}\sin(\varphi(t))\dot\varphi(t) \\
  \alpha_A_x(t) &= -l_1 \left(
      \cos(\theta(t))\dot{\theta}(t)^2 + \sin(\theta(t))\ddot{\theta}(t)
    \right) - l_2 \left(
      \cos(\varphi(t))\dot{\varphi}(t)^2 + \sin(\varphi(t))\ddot{\varphi}(t)
    \right)
\end{align}

\subsubsection{Composante $y$}

\begin{align}
  \alpha_{A_y}(t) &=
      \dd{t} l_1\cos(\theta(t))\cdot\dot{\theta}(t)
    + \dd{t} l_2\cos(\varphi(t))\cdot\dot{\varphi}(t) \\
  \alpha_{A_y}(t) &=
      l_1\dd{t} \cos(\theta(t))\cdot\dot{\theta}(t)
    + l_2\dd{t} \cos(\varphi(t))\cdot\dot{\varphi}(t) \\
  \alpha_{A_y}(t) &=
     l_1\left(-\sin(\theta(t))\cdot\dot\theta(t)\cdot\dot{\theta}(t) + \cos(\theta(t))\cdot\ddot{\theta}(t)\right)
   + l_2\dd{t} \cos(\varphi(t))\cdot\dot{\varphi}(t) \\
  \alpha_{A_y}(t) &=
     l_1\left(\cos(\theta(t))\cdot\ddot{\theta}(t) -\sin(\theta(t))\cdot\dot\theta(t)^2 \right)
   + l_2\dd{t} \cos(\varphi(t))\cdot\dot{\varphi}(t) \\
  \alpha_{A_y}(t) &=
     l_1\left(\cos(\theta(t))\cdot\ddot{\theta}(t) -\sin(\theta(t))\cdot\dot\theta(t)^2 \right)
   + l_2\left( -\sin(\varphi(t))\cdot\dot\varphi(t)\cdot\dot{\varphi}(t) + \cos(\varphi(t))\cdot\ddot{\varphi}(t) \right) \\
  \alpha_{A_y}(t) &=
     l_1\left(\cos(\theta(t))\cdot\ddot{\theta}(t) -\sin(\theta(t))\cdot\dot\theta(t)^2 \right)
   + l_2\left(\cos(\varphi(t))\cdot\ddot{\varphi}(t) -\sin(\varphi(t))\cdot\dot\varphi(t)^2 \right)
\end{align}
