
Dans un premier temps, il est demandé d'observer la force et le moment en
$B$ alors que le bras est immobile (statique) selon l'angle auquel il se
trouve. Un DSL n'analysant que le segment $\overline{BA}$ est utilisé pour
faire abstraction de ce qui se produit en ce qui concerne $\overline{OB}$ et
de ses forces, considérant le tout commes des forces/moments externes
nécessaires à garder la tige observée immobile.

\begin{multicols}{2}
  \begin{figure}[H]
    \centering
    \includegraphics[width=0.65\linewidth]{figures/static.pdf}
    \caption{DCL de la tige $\overline{BA}$}
    \label{fig:dcl}
  \end{figure}

  \columnbreak

	\begin{definitions}[]
    \vec{\alpha_g} & Accélération gravitationelle \\
    m_A            & Masse de l'objet accroché au point $A$ \\
    m_{BA}         & Masse de la tige $\overline{BA}$ \\
    \vec{P_l_2}    & Poids de la tige $l_2$ \\
    \vec{P_A}      & Poids de l'objet accroché au point $A$ \\
    \vec{P_B}      & Poids du moteur $B$ \\
    \vec{F_B}      & Force en $B$ nécessaire afin que la somme des forces soit nulle \\
    \vec{BA}       & Tige $\overline{BA}$ \\
    \vec{BA}/2     & Moitié de la tige $\overline{BA}$ \\
    \vec{M_l_2}    & Moment au point $B$ causé par $\vec{P_l_2}$ \\
    \vec{M_A}      & Moment au point $B$ causé par $\vec{P_A}$ \\
    \vec{M_B}      & Moment en $B$ nécessaire afin que la somme des moments nul
  \end{definitions}
\end{multicols}

\begin{align}
  \vec{P_l_2} &= m_{BA} \cdot \vec{\alpha_g}
    = -9.81m_{BA}\vec{j} \\
  \vec{P_A} &= m_A \cdot \vec{\alpha_g}
    = -9.81m_A\vec{j} \\
  \vec{P_B} &= m_B \cdot \vec{\alpha_g}
    = -9.81m_B\vec{j} \\
    \sum{\vec{F}} &= \vec{P_l_2} + \vec{P_A} + \vec{F_B} + \vec{P_B}= \vec{0} \\
  \vec{0} &= -9.81m_{BA}\vec{j} + (-9.81m_A\vec{j}) + \vec{F_B} + (-9.81m_B\vec{j}) \\
  \vec{F_B} &= 9.81m_{BA}\vec{j} + 9.81m_A\vec{j} + 9.81m_B\vec{j} \\
  \vec{F_B} &= 9.81(m_{BA}+m_A+m_B)\vec{j}
\end{align}

Les moments sont calculés en fonction du point $O$. Le moment causé par le poids
du moteur n'est pas considéré puisqu'il est nul (il est à une distance 0 du
point $O$).

\begin{align}
  \vec{BA}   &= \begin{bmatrix} l_2\cos(\varphi)\\ l_2\sin(\varphi)\\ 0 \end{bmatrix} \\
  \vec{BA}/2 &= \frac{1}{2} \begin{bmatrix} l_2\cos(\varphi)\\ l_2\sin(\varphi)\\ 0 \end{bmatrix}
    = \begin{bmatrix} l_2\cos(\varphi)/2 \\ l_2\sin(\varphi)/2 \\ 0 \end{bmatrix} \\
  \vec{M_A} &= \vec{BA}\times\vec{P_A}
    = \begin{bmatrix} l_2\cos(\varphi)\\ l_2\sin(\varphi)\\ 0 \end{bmatrix}
      \times (-9.81 m_A\vec{j}) \\
    &=  \vec{i} \begin{vmatrix} l_2\sin(\varphi) & 0         \\ -9.81m_A & 0        \end{vmatrix}
      - \vec{j} \begin{vmatrix} l_2\cos(\varphi) & 0         \\ 0        & 0        \end{vmatrix}
      + \vec{k} \begin{vmatrix} l_2\cos(\varphi) & l_2\sin(\varphi) \\ 0        & -9.81m_A \end{vmatrix} \\
    &= 0\vec{i} - 0\vec{j} + \left(-9.81m_Al_2\cos(\varphi)-0\right)\vec{k}
      = -9.81m_Al_2\cos(\varphi)\vec{k} \\
  \vec{M_l_2} &= (\vec{BA}/2)\times\vec{P_l_2}
    = \begin{bmatrix} l_2\cos(\varphi)/2 \\ l_2\sin(\varphi)/2\\ 0 \end{bmatrix}
      \times (-9.81 m_{BA}\vec{j}) \\
    &=  \vec{i} \begin{vmatrix} l_2\sin(\varphi)/2 & 0           \\ -9.81m_{BA} & 0          \end{vmatrix}
      - \vec{j} \begin{vmatrix} l_2\cos(\varphi)/2 & 0           \\ 0          & 0          \end{vmatrix}
      + \vec{k} \begin{vmatrix} l_2\cos(\varphi)/2 & l_2\sin(\varphi)/2 \\ 0          & -9.81m_{BA} \end{vmatrix} \\
    &= 0\vec{i} - 0\vec{j} + \frac{-9.81m_{BA}l_2\cos(\varphi)}{2}\vec{k}
      = -\frac{9.81m_{BA}l_2\cos(\varphi)}{2}\vec{k}
\end{align}
\begin{align}
  \sum{\vec{M}} &= \vec{M_A} + \vec{M_l_2} + \vec{M_B} = \vec{0} \\
  \vec{0} &= -9.81m_Al_2\cos(\varphi)\vec{k} + \left(-\frac{9.81m_{BA}l_2\cos(\varphi)}{2}\vec{k}\right) + \vec{M_B} \\
  \vec{M_B} &= 9.81m_Al_2\cos(\varphi)\vec{k} + \frac{9.81m_{BA}l_2\cos(\varphi)}{2}\vec{k} \\
    &= 9.81\cos(\varphi)l_2(m_A+m_{BA}/2)\vec{k}
\end{align}

Pour le cas statique, l'analyse porte sur l'évolution du couple résultant $M_{B_z}$ en fonction de l'angle $\varphi$ lorsque $\varphi$ varie entre $-\frac{\pi}{3}$ et $\frac{\pi}{3}$. La modélisation présentée à la figure \ref{fig:static-torque} utilise les valeurs spécifiées au tableau \ref{tab:values-statique-dynamique}. Dans ce cas, seules les forces gravitationnelles sont considérées, l'accélération angulaire étant nulle ($\ddot{\varphi} = 0$).

