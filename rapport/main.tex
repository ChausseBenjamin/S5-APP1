\documentclass[a11paper]{article}

% \usepackage{karnaugh-map}
\usepackage{tabularx}
\usepackage{titlepage}
\usepackage{amsmath}
\usepackage{amsthm}
\usepackage{document}
\usepackage{booktabs}
\usepackage{float}

\usepackage[dvipsnames]{xcolor}

% \renewcommand{\not}[1]{\overline{#1}}
\renewcommand{\not}[1]{#1^{\prime}}
\newcommand{\rhs}[1]{#1^{\prime}}
\newcommand{\xor}{\oplus}
\newcommand{\Cuz}[1]{\Arrow{\footnotesize #1}}

% \item $A+B=B+A \hspace{\fill}\text{commutativity}$
\usepackage{enumitem}       % customizable list environments


\title{Devoir}

\class{Logique Combinatoire}
\classnb{APP1}

\teacher{Marwan Besrour \& Gabriel Bélanger}

\newcommand{\todo}[1]{\begin{color}{Red}\textbf{TODO:} #1\end{color}}
\newcommand{\note}[1]{\begin{color}{Orange}\textbf{NOTE:} #1\end{color}}
\newcommand{\fixme}[1]{\begin{color}{Fuchsia}\textbf{FIXME:} #1\end{color}}
\newcommand{\question}[1]{\begin{color}{ForestGreen}\textbf{QUESTION:} #1\end{color}}

\newcommand{\dd}[1]{\frac{\partial}{\partial#1}}

\author{
  \addtolength{\tabcolsep}{-0.4em}
  \begin{tabular}{rcl} % Ajouter des auteurs au besoin
      Benjamin Chausse & -- & CHAB1704 \\
      Shawn Couture    & -- & COUS1912 \\
  \end{tabular}
}

\begin{document}
\maketitle


\section{Analyse cinématique}

\subsection{Positions}

\begin{align}
	\vec{XO} = \begin{bmatrix} 0\\l_0\\0 \end{bmatrix} \hspace{1cm}
	\vec{OB} &= \begin{bmatrix} l_1\cos{\theta}\\ l_1\sin{\theta} \\0 \end{bmatrix} \hspace{1cm}
	\vec{BA} = \begin{bmatrix} l_2\cos{\varphi}\\l_2\sin{\varphi} \\0 \end{bmatrix} \\
	\vec{B} = \vec{XO}+\vec{OB} &= \begin{bmatrix} l_1\cos{\theta}\\l_0+l_1\sin{\theta} \\0 \end{bmatrix} \\
	\vec{A} = \vec{XO}+\vec{OB} &= \begin{bmatrix}
	l_1\cos{\theta}+l_2\cos{\varphi}\\l_0+l_1\sin{\theta}+l_2\sin{\varphi} \\0 \end{bmatrix} \\
\end{align}

Autrement dit:

\begin{align}
	B_x(\theta) = l_1\cos{\theta} \hspace{1cm}
	  B_y(\theta) &= l_0+l_1\sin{\theta} \hspace{1cm}
		B_z(\theta) = 0 \\
	A_x(\theta,\varphi) = l_1\cos{\theta}+l_2\cos{\varphi} \hspace{1cm}
		A_y(\theta,\varphi) &= l_0+l_1\sin{\theta}+l_2\sin{\varphi} \hspace{1cm}
		A_z(\theta,\varphi) = 0
\end{align}

\subsection{Vitesses}


\begin{align}
	\theta \rightarrow \theta(t) &\hspace{1cm} \varphi \rightarrow \varphi(t)
\end{align}

\subsubsection{en relation à $x$}

\begin{align}
	A_x(t) &= l_1\cos{\theta(t)}+l_2\cos{\varphi(t)} \hspace{1cm} \\
	\dd{t} A_x(t) &= \dd{t} l_1\cos(\theta(t)) + \dd{t} l_2\cos(\varphi(t)) \\
	V_A_x(t) &= \dd{t} l_1\cos(\theta(t)) + \dd{t} l_2\cos(\varphi(t)) \\
	V_A_x(t) &= l_1\dd{t} \cos(\theta(t)) + l_2\dd{t} \cos(\varphi(t)) \\
	V_A_x(t) &= l_1\left( -\sin(\theta(t))\dot\theta(t)  \right) +
		l_2\left( - \sin(\varphi(t))\dot\varphi(t) \right) \\
	V_A_x(t) &= -l_1\sin(\theta(t))\dot\theta(t)
	  - l_2\sin(\varphi(t))\dot\varphi(t) \\
\end{align}

\subsection{Accélérations}

\subsubsection{en relation à $x$}

\begin{align}
	V_A_x(t) &= -l_1\sin(\theta(t))\dot\theta(t)
	  - l_2\sin(\varphi(t))\dot\varphi(t) \\
	\dd{t} V_A_x(t) &= \dd{t}\left(-l_1\sin(\theta(t))\dot\theta(t)
		- l_2\sin(\varphi(t))\dot\varphi(t)\right) \\
	\alpha_A_x(t) &= -l_1 \dd{t}\sin(\theta(t))\dot\theta(t)
	  - l_2\dd{t}\sin(\varphi(t))\dot\varphi(t) \\
	\alpha_A_x(t) &= -l_1 \left(
			\cos(\theta(t))\dot{\theta}(t)^2 + \sin(\theta(t))\ddot{\theta}(t)
		\right) - l_2 \left(
			\cos(\phi(t))\dot{\phi}(t)^2 + \sin(\phi(t))\ddot{\phi}(t)
		\right)
\end{align}

\end{document}
