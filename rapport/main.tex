\documentclass[a11paper]{article}

\usepackage{tabularx}
\usepackage{titlepage}
\usepackage{amsmath}
\usepackage{amsthm}
\usepackage{document}
\usepackage{booktabs}
\usepackage{float}
\usepackage{graphicx}
\usepackage{utils}
\usepackage{multicol}
\usepackage[dvipsnames]{xcolor}
\usepackage{enumitem}       % customizable list environments


\title{Rapport d'APP}

\class{Logique Combinatoire}
\classnb{APP1}

\teacher{Marwan Besrour \& Gabriel Bélanger}

\newcommand{\todo}[1]{\begin{color}{Red}\textbf{TODO:} #1\end{color}}
\newcommand{\note}[1]{\begin{color}{Orange}\textbf{NOTE:} #1\end{color}}
\newcommand{\fixme}[1]{\begin{color}{Fuchsia}\textbf{FIXME:} #1\end{color}}
\newcommand{\question}[1]{\begin{color}{ForestGreen}\textbf{QUESTION:} #1\end{color}}

\newcommand{\dd}[1]{\frac{\partial}{\partial#1}}
\renewcommand{\vec}[1]{\overrightarrow{#1}}

\author{
  \addtolength{\tabcolsep}{-0.4em}
  \begin{tabular}{rcl} % Ajouter des auteurs au besoin
      Benjamin Chausse & -- & CHAB1704 \\
      Shawn Couture    & -- & COUS1912 \\
  \end{tabular}
}

\begin{document}
\maketitle
\tableofcontents
\newpage


\section{Analyse cinématique}

\subsection{Analyse géométrique}
\label{sec:geometry}

\noindent
\begin{minipage}[t]{0.4\textwidth}
\begin{figure}[H]
  \centering
  \begin{center}
    \includegraphics[width=\textwidth]{figures/geometry.pdf}
  \end{center}
  \caption{Tiges du bras mécanique}
  \label{fig:geometry}
\end{figure}
\end{minipage}\hfill
\begin{minipage}[t]{0.6\textwidth}

\begin{align}
  \vec{XO} = \begin{bmatrix} 0\\l_0\\0 \end{bmatrix} \hspace{.5cm}
  \vec{OB} &= \begin{bmatrix} l_1\cos{\theta}\\ l_1\sin{\theta} \\0 \end{bmatrix} \hspace{.5cm}
  \vec{BA} = \begin{bmatrix} l_2\cos{\varphi}\\l_2\sin{\varphi} \\0 \end{bmatrix} \\
  \vec{B} = \vec{XO}+\vec{OB} &= \begin{bmatrix} l_1\cos{\theta}\\l_0+l_1\sin{\theta} \\0 \end{bmatrix} \\
  \vec{A} = \vec{B}+\vec{BA} &= \begin{bmatrix}
  l_1\cos{\theta}+l_2\cos{\varphi}\\l_0+l_1\sin{\theta}+l_2\sin{\varphi} \\0 \end{bmatrix} \\
  \intertext{Autrement dit:}
  B_x(\theta) &= l_1\cos{\theta} \\
  B_y(\theta) &= l_0+l_1\sin{\theta} \\
  A_x(\theta,\varphi) &= l_1\cos{\theta}+l_2\cos{\varphi} \\
  A_y(\theta,\varphi) &= l_0+l_1\sin{\theta}+l_2\sin{\varphi} \\
\end{align}
\end{minipage}

\subsection{Vitesses}
\label{sec:speed}

Lorsque présenté avec des angles fixes, il est possible de trouver les coordonés
$x$,$y$ en fonction de ceux-ci. Toutefois, il se peut que ces angles varie
dans le temps (ce qui est essentiel à considérer pour déterminer la vitesse du
point $A$). $\theta$ et $\varphi$ sont donc dorénavant présentés commes des fonctions
pour le reste de l'analyse cinématique générale du bras mécanique:

\begin{equation*}
  \theta \rightarrow \theta(t) &\hspace{1cm} \varphi \rightarrow \varphi(t)
\end{equation*}

Déterminer la vitesse du point $A$ ne devient par la suite qu'un exercice de
dérivation dans le temps de la position, en utilisant les équation dévelopées
en \ref{sec:geometry} comme fondement. Afin d'en faciliter la résolution (et
la lecture), cette opération est segmenté par composant $x$, $y$ dans les
sections suivantes.

\subsubsection{Composante $x$}

\begin{align}
  A_x(t) &= l_1\cos{\theta(t)}+l_2\cos{\varphi(t)} \hspace{1cm} \\
  \dd{t} A_x(t) &= \dd{t} l_1\cos(\theta(t)) + \dd{t} l_2\cos(\varphi(t)) \\
  V_A_x(t) &= \dd{t} l_1\cos(\theta(t)) + \dd{t} l_2\cos(\varphi(t)) \\
  V_A_x(t) &= l_1\dd{t} \cos(\theta(t)) + l_2\dd{t} \cos(\varphi(t)) \\
  V_A_x(t) &= l_1\left( -\sin(\theta(t))\dot\theta(t)  \right) +
    l_2\left( - \sin(\varphi(t))\dot\varphi(t) \right) \\
  V_A_x(t) &= -l_1\sin(\theta(t))\dot\theta(t)
    - l_2\sin(\varphi(t))\dot\varphi(t) \\
\end{align}

\subsubsection{Composante $y$}
\begin{align}
  A_y(t)&=l_0+l_1\sin(\theta(t))+l_2\sin(\varphi(t)) \\
  \dd{t}A_y(t) &=
      \dd{t}l_0
    + \dd{t}l_1\sin(\theta(t))
    + \dd{t}l_2\sin(\varphi(t)) \\
  V_{A_y}(t) &=
      \dd{t}l_1\sin(\theta(t))
    + \dd{t}l_2\sin(\varphi(t)) \\
  V_{A_y}(t) &=
      l_1\dd{t}\sin(\theta(t))
    + l_2\dd{t}\sin(\varphi(t)) \\
  V_{A_y}(t) &=
      l_1\cos(\theta(t))\cdot\dot{\theta}(t)
    + l_2\cos(\varphi(t))\cdot\dot{\varphi}(t)
\end{align}

\subsection{Accélérations}

L'accélération se définit comme étant la dérivée de la vitesse. Comme la
vitesse à déjà été définie de façon générale en \ref{sec:speed}, il ne sufit
qu'à en effectuer la dérivée (encore une fois par composante pour simplifier
la lecture) afin d'obtenir un solution générale de l'accélération selon
n'importe qu'elles fonction décrivant le mouvement angulaire des moteurs
dans le temps.

\subsubsection{Composante $x$}

\begin{align}
  V_A_x(t) &= -l_1\sin(\theta(t))\dot\theta(t)
    - l_2\sin(\varphi(t))\dot\varphi(t) \\
  \dd{t} V_A_x(t) &= \dd{t}\left(-l_1\sin(\theta(t))\dot\theta(t)
    - l_2\sin(\varphi(t))\dot\varphi(t)\right) \\
  \alpha_A_x(t) &= -l_1 \dd{t}\sin(\theta(t))\dot\theta(t)
    - l_2\dd{t}\sin(\varphi(t))\dot\varphi(t) \\
  \alpha_A_x(t) &= -l_1 \left(
      \cos(\theta(t))\dot{\theta}(t)^2 + \sin(\theta(t))\ddot{\theta}(t)
    \right) - l_2 \left(
      \cos(\varphi(t))\dot{\varphi}(t)^2 + \sin(\varphi(t))\ddot{\varphi}(t)
    \right)
\end{align}

\subsubsection{Composante $y$}

\begin{align}
  \alpha_{A_y}(t) &=
      \dd{t} l_1\cos(\theta(t))\cdot\dot{\theta}(t)
    + \dd{t} l_2\cos(\varphi(t))\cdot\dot{\varphi}(t) \\
  \alpha_{A_y}(t) &=
      l_1\dd{t} \cos(\theta(t))\cdot\dot{\theta}(t)
    + l_2\dd{t} \cos(\varphi(t))\cdot\dot{\varphi}(t) \\
  \alpha_{A_y}(t) &=
     l_1\left(-\sin(\theta(t))\cdot\dot\theta(t)\cdot\dot{\theta}(t) + \cos(\theta(t))\cdot\ddot{\theta}(t)\right)
   + l_2\dd{t} \cos(\varphi(t))\cdot\dot{\varphi}(t) \\
  \alpha_{A_y}(t) &=
		 l_1\left(\cos(\theta(t))\cdot\ddot{\theta}(t) -\sin(\theta(t))\cdot\dot\theta(t)^2 \right)
   + l_2\dd{t} \cos(\varphi(t))\cdot\dot{\varphi}(t) \\
  \alpha_{A_y}(t) &=
		 l_1\left(\cos(\theta(t))\cdot\ddot{\theta}(t) -\sin(\theta(t))\cdot\dot\theta(t)^2 \right)
	 + l_2\left( -\sin(\varphi(t))\cdot\dot\varphi(t)\cdot\dot{\varphi}(t) + \cos(\varphi(t))\cdot\ddot{\varphi}(t) \right) \\
  \alpha_{A_y}(t) &=
		 l_1\left(\cos(\theta(t))\cdot\ddot{\theta}(t) -\sin(\theta(t))\cdot\dot\theta(t)^2 \right)
	 + l_2\left(\cos(\varphi(t))\cdot\ddot{\varphi}(t) -\sin(\varphi(t))\cdot\dot\varphi(t)^2 \right)
\end{align}
\newpage


\section{Analyse statique}

  \begin{multicols}{2}
    \begin{definitions}[]
      \vec{\alpha_g} & Accélération gravitationelle \\
      m_A            & Masse de l'objet accroché au point $A$ \\
      m_{l_2}        & Masse de la tige $l_2$ \\
      \vec{P_l_2}    & Poids de la tige $l_2$ \\
      \vec{P_A}      & Poids de l'objet accroché au point $A$ \\
      \vec{F_I}      & Force Inconnu nécessaire afin que la somme des forces soit nulle \\
      \vec{BA}       & Vecteur de longueur de la tige $l_2$ \\
      \vec{BA}/2     & Vecteur de demi-longueur (centre) de la tige $l_2$ \\
      \vec{M_l_2} & Moment au point $B$ causé par la force $P_A$ \\
      \vec{M_A} & Moment au point $B$ causé par la force $P_A$ \\
      \vec{M_I} & Moment inconnu nécessaire afin que la somme des moments nul
    \end{definitions}

    \columnbreak

    \begin{figure}[H]
      \centering
      \includegraphics[width=0.6\linewidth]{figures/dcl.pdf}
      \caption{DCL de la tige $\overline{BA}$}
      \label{fig:dcl}
    \end{figure}
  \end{multicols}

\begin{align}
  \vec{P_l_2} &= m_l_2 \cdot \vec{\alpha_g}
    = m_l_2 \cdot \begin{bmatrix} 0 \\ -9.81 \\ 0 \end{bmatrix}
    = \begin{bmatrix} 0 \\ -9.81 m_l_2 \\ 0 \end{bmatrix} \\
  \vec{P_A} &= m_A \cdot \vec{\alpha_g}
    = m_A \cdot \begin{bmatrix} 0 \\ -9.81 \\ 0 \end{bmatrix}
    = \begin{bmatrix} 0 \\ -9.81 m_A \\ 0 \end{bmatrix} \\
  \sum{\vec{F}} &= \vec{P_l_2} + \vec{P_A} + \vec{F_I} = \vec{0} \\
  \vec{0} &=
    \begin{bmatrix} 0 \\ -9.81 m_l_2 \\ 0 \end{bmatrix} +
    \begin{bmatrix} 0 \\ -9.81 m_A \\ 0 \end{bmatrix} +
    \vec{F_I} \\
  \vec{F_I} &=
    - \begin{bmatrix} 0 \\ -9.81 m_l_2 \\ 0 \end{bmatrix}
    - \begin{bmatrix} 0 \\ -9.81 m_A \\ 0 \end{bmatrix} =
    \begin{bmatrix} 0 \\ 9.81 m_l_2 \\ 0 \end{bmatrix}  +
    \begin{bmatrix} 0 \\ 9.81 m_A \\ 0 \end{bmatrix} \\
  \vec{F_I} &=
    \begin{bmatrix} 0 \\ 9.81(m_l_2+ m_A) \\ 0 \end{bmatrix} \\
\end{align}

\begin{align}
  \vec{BA}   &= \begin{bmatrix} \cos{l_2}\\ \sin{l_2}\\ 0 \end{bmatrix} \\
  \vec{BA}/2 &= \frac{1}{2} \begin{bmatrix} \cos{l_2}\\ \sin{l_2}\\ 0 \end{bmatrix}
    = \begin{bmatrix} \cos{l_2}/2 \\ \sin{l_2}/2 \\ 0 \end{bmatrix} \\
\end{align}

\begin{align}
  \vec{M_A} &= \vec{BA}\times\vec{P_A}
    = \begin{bmatrix} \cos{l_2}\\ \sin{l_2}\\ 0 \end{bmatrix}
      \times \begin{bmatrix} 0 \\ -9.81 m_A \\ 0 \end{bmatrix} \\
    &=  \vec{i} \begin{vmatrix} \sin{l_2} & 0         \\ -9.81m_A & 0        \end{vmatrix}
      - \vec{j} \begin{vmatrix} \cos{l_2} & 0         \\ 0        & 0        \end{vmatrix}
      + \vec{k} \begin{vmatrix} \cos{l_2} & \sin{l_2} \\ 0        & -9.81m_A \end{vmatrix} \\
    &= 0\vec{i} - 0\vec{j} + \left(-9.81m_A\cos{l_2}-0\right)\vec{k}
      = \begin{bmatrix} 0\\0\\-9.81m_A\cos{l_2}\end{bmatrix} \\
  \vec{M_l_2} &= (\vec{BA}/2)\times\vec{P_A}
    = \begin{bmatrix} \cos{l_2}/2 \\ \sin{l_2}/2\\ 0 \end{bmatrix}
      \times \begin{bmatrix} 0 \\ -9.81 m_l_2 \\ 0 \end{bmatrix} \\
    &=  \vec{i} \begin{vmatrix} \sin{l_2}/2 & 0           \\ -9.81m_l_2 & 0          \end{vmatrix}
      - \vec{j} \begin{vmatrix} \cos{l_2}/2 & 0           \\ 0          & 0          \end{vmatrix}
      + \vec{k} \begin{vmatrix} \cos{l_2}/2 & \sin{l_2}/2 \\ 0          & -9.81m_l_2 \end{vmatrix} \\
    &= 0\vec{i} - 0\vec{j} + \frac{-9.81m_l_2\cos{l_2}}{2}\vec{k}
      = \begin{bmatrix} 0\\0\\-(9.81m_l_2\cos{l_2})/2 \end{bmatrix} \\
  \sum{\vec{M}} &= \vec{M_A} + \vec{M_l_2} + \vec{M_I} = \vec{0} \\
  \vec{0} &= \begin{bmatrix} 0\\0\\-9.81m_A\cos{l_2}\end{bmatrix}
    + \begin{bmatrix} 0\\0\\-(9.81m_l_2\cos{l_2})/2 \end{bmatrix}
    + \vec{M_I} \\
  \vec{M_I} &= - \begin{bmatrix} 0\\0\\-9.81m_A\cos{l_2}\end{bmatrix}
    - \begin{bmatrix} 0\\0\\-(9.81m_l_2\cos{l_2})/2 \end{bmatrix} \\
    &= \begin{bmatrix} 0\\0\\9.81m_A\cos{l_2}\end{bmatrix}
      + \begin{bmatrix} 0\\0\\(9.81m_l_2\cos{l_2})/2 \end{bmatrix} \\
        &= \begin{bmatrix} 0\\0\\9.81(m_A\cos{l_2}+m_l_2\cos{l_2}/2) \end{bmatrix}
\end{align}

\end{document}
