\subsection{Mouvement contraint horizontalement}
\label{sec:x-bound}
\todo{raisonnement qui nous ammène à $\varphi=-\theta$}

Ainsi, puisque dans ce cas d'étude spécifique, $\varphi = -\theta$, les
solutions trouvées en \ref{sec:geometry} peuvent être transformées comme suit:

\begin{align}
  A_x(\theta,\varphi) = l_1\cos{\theta}+l_2\cos{\varphi} &\Rightarrow
  A_x(\theta)         = l_1\cos{\theta}+l_2\cos(-\theta)
\end{align}

Avec peu, d'effort, ce même type de transformation peut être fait pour les
équation de vitesses et d'accélération où $\theta$ dépend du temps.
Puisque $-1$ est un constante, la dérivation est triviale:

\begin{equation}
  \varphi(t)        = -\theta(t) \Rightarrow
  \dot{\varphi}(t)  = -\dot{\theta}(t)  \Rightarrow
  \ddot{\varphi}(t) = -\ddot{\theta}(t)
\end{equation}

Cela permet d'exprimer les équation de vitesse et d'accélération de $A$ comme
suit:

\begin{align}
  V_A_x(t) &= -l_1\sin(\theta(t))\dot\theta(t)
    + l_2\sin(-\theta(t))\dot\theta(t) \\
  \alpha_A_x(t) &= -l_1 \left(
      \cos(\theta(t))\dot{\theta}(t)^2 + \sin(\theta(t))\ddot{\theta}(t)
    \right) - l_2 \left(
      -\cos(-\theta(t))\dot{\theta}(t)^2 - \sin(-\theta(t))\ddot{\theta}(t)
    \right)
\end{align}

Dans le câdre de cette analyse, il est demandé de modéliser le bras entre
$\theta=0$ et $\theta=\frac{\pi}{3}$ posant que la vitesse angulaire de
$\theta$ est constante et vaut $\omega$. Puisque $\dot{\theta}(t)$ est en soit
une accélération angulaire, nous obtenons ceci:

\begin{equation}
  \dot{\theta}(t) = \omega \Rightarrow
    \ddot{\theta}(t) = 0 \Rightarrow
		\label{eq:simple-derivatives}
    \theta(t) = \omega t \\
\end{equation}

\begin{align}
	A_x(t)   &= l_1\cos(\theta(t))+l_2\cos(-\theta(t)) \Rightarrow
	A_x(t)   = l_1\cos(\omega t)+l_2\cos(-\omega t) \\
  V_A_x(t) &= -l_1\sin(\omega t)\omega
    + l_2\sin(-\omega t)\omega \\
  \alpha_A_x(t) &= -l_1 \left( \cos(\omega t )\omega^2 + \sin(\omega t )\cdot 0 \right)
    - l_2 \left( -\cos(-\omega t)\omega^2 - \sin(-\omega t )\cdot 0 \right) \\
    &=
    - l_1 \cos(\omega t )\omega^2
    + l_2 \cos(-\omega t)\omega^2
\end{align}

Toutefois l'analyse de ce cas porte sur les valeurs de
position/vitesse/accélérations selon $\theta$ et non pas selon $t$.
Les deux étant interdépendants les équation ci-dessus peuvent être mises
selon $\theta$ puisque $\theta = \omega t$ implique que:

\begin{equation}
	\label{eq:substitute-time}
	t = \frac{\theta}{\omega}
\end{equation}
Ainsi:

\begin{align}
	A_x(t)   &= l_1\cos(\omega \frac{\theta}{\omega})+l_2\cos(-\omega \frac{\theta}{\omega}) \\
	         &= l_1\cos(\theta)+l_2\cos(-\theta) \\
  V_A_x(\theta) &= -l_1\sin(\omega \frac{\theta}{\omega})\omega
    + l_2\sin(-\omega \frac{\theta}{\omega})\omega \\
           &= -l_1\sin(\theta)\omega
    + l_2\sin(-\theta)\omega \\
  V_{A_y}(\theta) &=
      l_1\cos(\omega \frac{\theta}{\omega})\omega
    - l_2\cos(-\omega \frac{\theta}{\omega})\omega \\
    &=
      l_1\cos(\theta)\omega
    - l_2\cos(-\theta)\omega \\
  \alpha_{A_y}(\theta) &=
    - l_1 \sin(\omega \frac{\theta}{\omega})\omega^2
    + l_2\sin(-\omega \frac{\theta}{\omega})\omega^2 \\
    &=
    - l_1 \sin(\theta)\omega^2
    + l_2\sin(-\theta)\omega^2 \\
  \alpha_A_x(\theta) &=
    - l_1 \cos(\omega \frac{\theta}{\omega} )\omega^2
    + l_2 \cos(-\omega \frac{\theta}{\omega})\omega^2 \\
    &=
    - l_1 \cos(\theta)\omega^2
    + l_2 \cos(-\theta)\omega^2
\end{align}

Pour le cas à l'étude $0\leq\theta\leq\frac{\pi}{3}$, une modélisation des
équation ci-dessus peut être consulté en annexe (Figure \ref{fig:x-bound}).

