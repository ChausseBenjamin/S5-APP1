\subsection{Mouvement contraint verticalement}

Dans ce cas de figure, la position en $x$ du point $A$ est définie comme étant
toujours à une distance $l$ de l'origine. D'ailleurs $l$ correspond aussi à la
longueur de $l_1$ et $l_2$. L'équation générale de $A_x$ peut être utilisé pour
contraindre $\varphi$ en fonction de $\theta$:

\begin{align}
  A_x = l_1\cos{\theta}+l_2\cos{\varphi} \Rightarrow
  l &= l\cos{\theta}+l\cos{\varphi} \\
  \label{eq:vert-simple-x}
  1 &= \cos{\theta}+\cos{\varphi} \\
  1-\cos{\theta} &= \cos{\varphi} \\
  \varphi &= \arccos(1-\cos{\theta})
\end{align}

\newpage
Pour éviter d'avoir à dériver $\varphi$ dans le temps, il est possible
d'utiliser l'équation \ref{eq:vert-simple-x} et les dérivées partielles:

\begin{align}
  1 &= \cos{\theta}+\cos{\varphi} \\
  \dd{t} 1 &= \dd{t}\left(\cos{\theta}+\cos{\varphi}\right) \\
         0 &= \dd{t}\cos{\theta}+\dd{t}\cos{\varphi} \\
         0 &= -\sin(\theta)\dot{\theta}+\dd{t}\cos{\varphi} \\
         0 &= -\sin(\theta)\dot{\theta}-\sin(\varphi)\dot{\varphi} \\
         \sin(\varphi)\dot{\varphi} &= -\sin(\theta)\dot{\theta} \\
         \dot{\varphi} &= -\frac{\sin(\theta)}{\sin(\varphi)}\dot{\theta}
\end{align}

Enfin puisque le mouvement est contraint, par définition, à un mouvement que sur
l'axe des $y$, il est seulement nécessaire d'analyser la position et la vitesse
selon cet axe. Encore une fois, les équation générales sont utilisées comme
point de départ:

\begin{align}
	A_y(t) &= l+l\sin{\theta(t)}+l\sin{\varphi(t)} \\
         &= l+l\sin{\theta(t)}+l\sin{\arccos(1-\cos(\theta(t)))} \\
         &= l+l\sin{\theta(t)}+l\sin{\arccos(1-\cos(\theta(t)))} \\
				 &= l\left[1+\sin{\theta(t)}+\sin{\arccos(1-\cos(\theta(t)))}\right] \\
  V_{A_y}(t) &=
      l_1\cos(\theta(t))\dot{\theta}(t)
    + l_2\cos(\varphi(t))\dot{\varphi}(t) \\
             &=
    l\cos(\theta(t))\cdot\dot{\theta}(t) +
    l\cos(\varphi(t))\left(-\frac{\sin(\theta(t))}{\sin(\varphi(t))}\dot{\theta}(t)\right) \\
             &=
      l\cos(\theta(t))\cdot\dot{\theta}(t)
    - l\cos(\arccos(1-\cos(\theta(t)))) \left(
        \frac{\sin(\theta(t))}{\sin(\arccos(1-\cos(\theta(t))))}\dot{\theta}(t)
      \right) \\
\end{align}

Puisque la vitesse angulaire de $\theta$ est constant est la même que dans le
cas d'étude horizontal, il est possible d'utiliser la relation de l'équation
\ref{eq:simple-derivatives} pour éliminer tout $\dot{\theta}(t)$ et
enfin tout considérer en fonction de $\theta$ au lieu du temps à l'aide de
l'équation \ref{eq:substitute-time}:

\begin{align}
	A_y(t)     &= l\left[1+\sin(\omega t)+\sin(\arccos(1-\cos(\omega t)))\right] \\
	A_y(\theta)&= l\left[1+\sin(\omega t)+\sin(\arccos(1-\cos(\omega t)))\right] \\
	           &= l\left[1+\sin(\omega \frac{\theta}{\omega})+\sin(\arccos(1-\cos(\omega \frac{\theta}{\omega})))\right] \\
	           &= l\left[1+\sin(\theta)+\sin(\arccos(1-\cos(\theta)))\right]
\end{align}
\begin{align}
  V_{A_y}(t) &=
      l\cos(\omega t)\omega
    - l\cos(\arccos(1-\cos(\omega t))) \left(
        \frac{\sin(\omega t)}{\sin(\arccos(1-\cos(\omega t)))}\omega
      \right) \\
  V_{A_y}(\theta) &=
			l\cos(\omega \frac{\theta}{\omega})\omega
		- l\cos(\arccos(1-\cos(\omega \frac{\theta}{\omega}))) \left(
				\frac{\sin(\omega \frac{\theta}{\omega})}{\sin(\arccos(1-\cos(\omega
				\frac{\theta}{\omega})))}\omega
      \right) \\
                  &=
			l\cos(\theta)\omega
		- l\cos(\arccos(1-\cos\theta)) \left(
				\frac{\sin\theta}{\sin(\arccos(1-\cos\theta))}\omega
      \right) \\
                  &=
			l\cos(\theta)\omega
		- l\cos(\arccos(1-\cos\theta)) \left(
				\frac{\sin\theta}{\sin(\arccos(1-\cos\theta))}\omega
      \right) \\
                  &=
			l\left[\cos(\theta)\omega
		- \cos(\arccos(1-\cos\theta)) \left(
				\frac{\sin\theta}{\sin(\arccos(1-\cos\theta))}\omega
			\right)\right]
\end{align}

Pour le cas à l'étude $0\leq\theta\leq\frac{\pi}{3}$, une modélisation des
équations (figure \ref{fig:vertical-plots}) avec les valeurs spécifiés au tableau
\ref{tab:values-cinematique} peut être trouvée en annexe. Aussi, les
positions de départ et de fin du bras pour cette modélisation peuvent être
consulté à la figure \ref{fig:vertical-fig}.

